\section{Design}

\todo{
    \begin{itemize}
        \item sec overview
        \item the \name{} client library
            \begin{itemize}
            \item functionality: sharing and encryption
            \item data types \& invariants
            \item groups-device groups, social groups
            \end{itemize}
        \item the \name{} server
        \item \name{} applications
    \end{itemize}
}


\subsection{Server}
The server is responsible for forwarding messages and imposing a global 
order on all routed messages. \dots can be achieved using a database with
serializable transactions \dots

\name{} enables developers to easily build secure applications. It does
so by exposing a simple interface that handles interactions with the
application data store and with other clients. All application data is 
stored in on-device local storage, and all cross-client communication is \
encrypted. Thus, no private information is exposed or persisted on a 
central server or database, ensuring user privacy. 

The two main components of the \name{} architecture are the 
\textit{client library} and the \textit{\name{} server}. All \name{} 
applications utilize the library's exposed functions to enable 
off-device communication and consistency for any shared data. The server
facilitates communication across clients while enforcing a global order 
on messages, and is application-agnostic, meaning it does not need to be 
modified by application developers and service multiple applications 
simultaneously. 

\subsection{Client-side Library}
\dots \fix{talk about the api that is exposed, local-only application reacting to a local data store} 
application developers get cross-client communication "for free",
something about local data store, etc \dots
\fix{firebase, reactDB}

\todo{
\begin{itemize}
    \item what is the api
    \item what is simple sharing
    \item encryption
    \item groups
    \item app invariants -> permissions
\end{itemize}
}


\parlabel{Data}
The \name{} client-side library facilitates application interactions with
a local storage mechanism, e.g. localStorage in browsers, sqlite on 
smartphones. Applications read and write to storage as usual, but or 
any shared data, the library will track the change and propagate them to
other clients. Likewise, any remote changes recieved from other clients
will be applied automatically by the library.

Fascilitating interactions with the underlying data storage mechanism
allows \name{} applications to check data invariants before applying 
changes.
\todo{maybe in impl -> what is the interface for defining data invariants}
  
\parlabel{Sharing}
When a change to any data object is initiated on a client, the client
then enumerates a list of all clients the change must be propagated to
based on the group assocuated with the data object.
The client then encrypts the message outlining the change for each 
destination device, and posts it to the server. The server forwards
each encrypted message to the appropriate destination clients.

\parlabel{Groups}
All data object are associated with a \textit{group} defining access 
permissions for the specific data object. A group is a list of pointers 
to any combination of other groups, user groups, or devices. Any group
can be recursively enumerated to a list of destination clients. 
\fix{self-group upon initialization}
\fix{talk about data structure a little more}
\fix{?posix groups?}

Changes to the group field of an object are tracked the same way as 
changes to any other object subfield, and propagated similarly to 
relevant clients. Groups are treated as \name{} native data objects,
with the group itself as the permission field.  

\todo{permissions}

\parlabel{Encryption}








