\section{Background and Related Work}

\paragraph{Private messaging.}
Signal \cite{signal} establishes a privacy protocol for instant messaging, in which messages between users are end-to-end encrypted with additional session keys to achieve forward secrecy in the presence of compromised keys. Group information is stored encrypted on the server, but the server is not entirely oblivious to all group information as it "special". The server also knows which devices belong to which users (a kind of group). Signal leverages other specific properties about messaging in their highly-specific protocol, such as weaker consistency needs and simple application invariants (messages should be appended).

Whatsapp \cite{whatsapp}.
Matrix \cite{matrix}.

Stadium \cite{stadium}. \todo{different subsection for different flavor of private messaging.}

\paragraph{Encryption.} \todo{maybe combine with private messaging subsection, although still good to have some separation between encryption and rest of protocol.}
Signal \cite{signal} and Matrix \cite{matrix} both provide a set of end-to-end 
encrypted messaging APIs for private, decentralized communication in an 
asynchronous settings (clients 
can be intermittently offline). Both protocols achieve forward secrecy (where
the confidentiality of older messages is preserved in the face of a compromised
key) 
\todo{and backwards secrecy?}.

They achieve this through the double ratchet algorithm \cite{}, in which
two clients each start with a shared secret key (established through an initial
key-agreement protocol). An initiator client computes the initial shared secret 
key using it's own public/private keys and the public (possibly-signed) key
counterparts of the recipient, and sends an initial message to the recipient
encrypted with the secret key (or some derivation of it). When both the initiator
and recipient are able to compute this shared secret key, a double ratchet session
has started. Then each subsequent message sent using this session is encrypted with
a new key derived from the shared secret key, these keys forming a chain of keys
such that a compromised key somewhere in that chain does not affect the 
confidentiality of messages encrypted with earlier or later keys. \todo{Describe
double ratchet in a bit more detail.} A double
ratchet session exists for every pair of devices \todo{and separately, every
group?}, and all sessions must be appropriately managed. 

An example
of session-management complexity is when the initial message is received by the
recipient but not replied to. A session is technically created on the recipient
but the sender must assume that it was not, because it was never confirmed. 
Therefore, if the sender wants to send another message to the recipient, it must
re-execute the key-agreement algorithm and generate a new session. This process
is repeated until the recipient replies to the sender under one of the sessions.
\todo{more broad session management stuff?}

Differences between the two \todo{need to synthesize this more}:
\begin{enumerate}
\item Signal implements \textit{extended} triple diffie-hellman for it's key agreement
algorithm, which uses slightly different/more keys than Olm's (Matrix's crypto 
lib) implementation of triple diffie-hellman (not extended).
Olm uses one identity key 
and one one-time key per client (where the recipient's 
one-time key is published to the server and obtained by the
initiator). Signal uses one identity key per client, an 
ephemeral key for the initiating client (which does not leave
the client), and a bundle published to the server containing
a signed prekey and a one-time key for the receiving client.
\temp{Signal's one-time key seems to be optional, maybe if high load and the receiving client cannot upload quickly enough, this could help with performance. Have not yet found an actualy explanation this is just my guess.}
All identity keys (in both protocols) are long-term and are published to the server.
\item Olm uses authenticated encryption. Signal does not seem to use authenticated 
encryption, but does require that one of the keys in the pre-key bundle be 
signed. 
\item Signal allows client to either use the secret
key directly to encrypt the first message, or another key
derived from it (via a perfectly-random-function). Olm always encrypts the first
message with a key derived from the secret key (the message key).
\item The contents of the first message are slightly different.
\item When a session is established, Olm generates three additional keys: a root, 
chain, and ratchet key (in addition to the message key). Signal also generates
three keys by different names (\todo{check overlaps in meaning}): a root, 
sender, and receiving chain.
\end{enumerate}

\paragraph{Untrusted servers.}
SPORC \cite{sporc} presents a similar application model that primarily leverages the server to establish a global order of events and remaining functionality is handled on the client. However, the SPORC server has a predefined notion of groups and any group modifications are visible to the server, whereas the \name server is completely oblivious to any notion of groups and treats them just like all other data. SPORC also focuses on "operational transform" applications in which operations are meaningful regardless of the order they are applied in. This property is not the case for most applications (and most applications that \name supports), where some operations can be invalidated by others. Furthermore, SPORC does not generalize their application model, although they hint that it could be done, nor does it examine how application invariants could be maintained in the face of malicious or buggy clients. 

SUNDR \cite{sundr}.
Depot \cite{depot}.
Mylar \cite{mylar}.
CryptDB \cite{cryptdb}.

\paragraph{Peer to peer systems.}

Bayou \cite{}.
